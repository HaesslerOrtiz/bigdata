\documentclass[12pt,a4paper]{article}

% IMPORTACIÓN PAQUETES
% Acentos
\usepackage[utf8]{inputenc}
\usepackage[T1]{fontenc}
% Idioma español
\usepackage[spanish]{babel}
% Insertar imágenes
\usepackage{graphicx}
% Subfiguras
\usepackage{subcaption}
% Espaciado
\usepackage{setspace}
% Comillas inteligentes
\usepackage{csquotes}
% Márgenes
\usepackage{geometry}
\geometry{margin=2.5cm}
% Hipervínculos
\usepackage{hyperref}
\hypersetup{
    colorlinks=true,
    linkcolor=black,
    citecolor=black,
    urlcolor=black
}
% Referencias inteligentes
\usepackage{cleveref}

% Configuración de cleveref en español
\crefname{figure}{figura}{figuras}
\Crefname{figure}{Figura}{Figuras}
\crefname{subfigure}{figura}{figuras}
\Crefname{subfigure}{Figura}{Figuras}

% Conjunciones en español
\AtBeginDocument{\renewcommand{\creflastconjunction}{ y~}}
\AtBeginDocument{\renewcommand{\crefrangeconjunction}{ a~}}
\AtBeginDocument{\renewcommand{\crefmiddleconjunction}{, }}

% Para no repetir la carpeta de resultados en cada imagen
\graphicspath{{resultados/}}

% Título y autoría
\title{\textbf{Análisis exploratorio de datos del Dataset "EVA" del IDEAM}}
\author{Haessler Joan Ortiz Moncada \\[0.5cm]
        Universidad Distrital Francisco José de Caldas \\
        Facultad de Ingeniería \\
        Curso: BIG DATA}
\date{\today}

\begin{document}

% Título
\maketitle

% Descripción del dataset
\section{Descripción del Dataset}
El conjunto de datos \textit{EDA}, proporcionado por el IDEAM (Instituto de Hidrología, Meteorología y Estudios Ambientales de Colombia), 
contiene información sobre los cultivos agrícolas registrados en diferentes municipios de Colombia entre los años 2006 y 2018. 
En total, el dataset incluye aproximadamente 206\,068 registros y diecisiete variables, las cuales se describen en la \cref{tab:campos}:

\begin{table}[]
    \centering
    \begin{tabular}{|l|l|l|}
        \hline
        \multicolumn{1}{|c|}{\textbf{Nombre Campo}} & \multicolumn{1}{c|}{\textbf{Significado}}     & \multicolumn{1}{c|}{\textbf{Tipo}} \\ \hline
        cod\_dpto                                   & Código departamento                           & int64                              \\ \hline
        departamento                                & Nombre departamento                           & str                                \\ \hline
        cod\_mun                                    & Código municipio                              & int64                              \\ \hline
        municipio                                   & Nombre municipio                              & str                                \\ \hline
        gr\_cult                                    & Grupo cultivo                                 & str                                \\ \hline
        Sub\_gr\_cult                               & Subgrupo cultivo                              & str                                \\ \hline
        cultivo                                     & Cultivo                                       & str                                \\ \hline
        des\_reg\_sist\_prod                        & Desagregación regional y/o sistema productivo & str                                \\ \hline
        year                                        & Año                                           & int64                              \\ \hline
        prd                                         & Periodo                                       & str                                \\ \hline
        ha\_semb                                    & Área sembrada (ha)                            & int64                              \\ \hline
        ha\_csda                                    & Área cosechada (ha)                           & int64                              \\ \hline
        t\_prod                                     & Produccion (t)                                & int64                              \\ \hline
        t\_ha\_rend                                 & Rendimiento (t/ha)                            & float64                            \\ \hline
        st\_fis\_prd                                & Estado físico de producción                   & str                                \\ \hline
        name\_st                                    & Nombre científico                             & str                                \\ \hline
        cl\_clt                                     & Ciclo de cultivo                              & str                                \\ \hline
    \end{tabular}
    \caption{Campos del dataset EDA del IDEAM}
    \label{tab:campos}
\end{table}

% 
\section{Estadisticas básicas}
Como forma de entender mejor el dataset, se realizó un análisis exploratorio de datos (EDA, por sus siglas en inglés)
mediante la clase \texttt{Eda} desarrollada por mi e implementada en Python. Esta clase permite contar los registros, 
obtener los nombres y tipos de los campos, generar histogramas y diagramas de caja para variables numéricas,
y crear tablas de contingencia para variables categóricas. Antes de presentar esta infromación, es importante mencionar 
que la clase incluye un método de limpieza de datos que elimina los registros con valores faltantes o inconsistentes, lo
que redujo el número total de registros de 206\,068 a 199\,801. A continuación, se presentan algunas estadísticas básicas 
obtenidas del dataset:

\begin{center}
    \begin{itemize}
        \item \textbf{Variable}: ha\_semb
        \begin{itemize}
            \item mean: 298.38
            \item std: 1169.24
            \item min: 0.00
            \item 25\%: 10.00
            \item 50\%: 38.00
            \item 75\%: 160.00
            \item max: 47\,403.00
        \end{itemize}

        \item \textbf{Variable}: ha\_csda
        \begin{itemize}
            \item mean: 256.93
            \item std: 994.66
            \item min: 0.00
            \item 25\%: 9.00
            \item 50\%: 30.00
            \item 75\%: 137.00
            \item max: 38\,600.00
        \end{itemize}

        \item \textbf{Variable}: t\_prod
        \begin{itemize}
            \item mean: 2\,874.78
            \item std: 45\,814.01
            \item min: 0.00
            \item 25\%: 36.00
            \item 50\%: 150.00
            \item 75\%: 4\,546.00
            \item max: 123\,000.00
        \end{itemize}

        \item \textbf{Variable}: t\_ha\_rend
        \begin{itemize}
            \item mean: 9.27
            \item std: 14.96
            \item min: 0.03
            \item 25\%: 1.50
            \item 50\%: 5.00
            \item 75\%: 11.50
            \item max: 246.00
        \end{itemize}
    \end{itemize}
\end{center}

Lógicamente, estas estadísticas solo son representativas para las variables numéricas del dataset. 
Estas estadísiticas permiten ver que existe una alta variabilidad en los datos, sin embargo, esto 
se explica por la presencia de outliers positivos altos, más no porque la varibilidad general 
de los datos sea alta.

Para las variables categóricas, se optó por mostrar una tabla de contingencia entre las variables 
\textbf{departamento} y \textbf{Sub\_gr\_cult} (y solo para AGUACATE, BANANO y CAFÉ), la cual se 
presenta en la \cref{tab:contingencia}. Recordar que el presente informe no busca hacer un análisis 
exhaustivo, sino más bien ilustrar cómo realizar un Análisis Exploratorio de Datos (EDA por sus siglas 
en inglés) básico.


\begin{table}[]
    \centering
    \begin{tabular}{|llll|}
        \hline
        \multicolumn{4}{|c|}{\textbf{Tabla de contingencia (filtrada)}}                                                                                \\ \hline
        \multicolumn{1}{|l|}{\textbf{Departamento}}    & \multicolumn{1}{l|}{\textbf{AGUACATE}} & \multicolumn{1}{l|}{\textbf{BANANO}} & \textbf{CAFE} \\ \hline
        \multicolumn{1}{|l|}{AMAZONAS}                 & \multicolumn{1}{l|}{8}                 & \multicolumn{1}{l|}{2}               & 0             \\ \hline
        \multicolumn{1}{|l|}{ANTIOQUIA}                & \multicolumn{1}{l|}{674}               & \multicolumn{1}{l|}{286}             & 1096          \\ \hline
        \multicolumn{1}{|l|}{ARAUCA}                   & \multicolumn{1}{l|}{28}                & \multicolumn{1}{l|}{0}               & 1             \\ \hline
        \multicolumn{1}{|l|}{ATLANTICO}                & \multicolumn{1}{l|}{2}                 & \multicolumn{1}{l|}{0}               & 0             \\ \hline
        \multicolumn{1}{|l|}{BOLIVAR}                  & \multicolumn{1}{l|}{89}                & \multicolumn{1}{l|}{0}               & 20            \\ \hline
        \multicolumn{1}{|l|}{BOYACA}                   & \multicolumn{1}{l|}{159}               & \multicolumn{1}{l|}{62}              & 471           \\ \hline
        \multicolumn{1}{|l|}{CALDAS}                   & \multicolumn{1}{l|}{271}               & \multicolumn{1}{l|}{62}              & 300           \\ \hline
        \multicolumn{1}{|l|}{CAQUETA}                  & \multicolumn{1}{l|}{4}                 & \multicolumn{1}{l|}{0}               & 81            \\ \hline
        \multicolumn{1}{|l|}{CASANARE}                 & \multicolumn{1}{l|}{64}                & \multicolumn{1}{l|}{11}              & 85            \\ \hline
        \multicolumn{1}{|l|}{CAUCA}                    & \multicolumn{1}{l|}{125}               & \multicolumn{1}{l|}{52}              & 380           \\ \hline
        \multicolumn{1}{|l|}{CESAR}                    & \multicolumn{1}{l|}{182}               & \multicolumn{1}{l|}{13}              & 228           \\ \hline
        \multicolumn{1}{|l|}{CHOCO}                    & \multicolumn{1}{l|}{29}                & \multicolumn{1}{l|}{165}             & 12            \\ \hline
        \multicolumn{1}{|l|}{CORDOBA}                  & \multicolumn{1}{l|}{10}                & \multicolumn{1}{l|}{0}               & 0             \\ \hline
        \multicolumn{1}{|l|}{CUNDINAMARCA}             & \multicolumn{1}{l|}{268}               & \multicolumn{1}{l|}{179}             & 817           \\ \hline
        \multicolumn{1}{|l|}{GUAINIA}                  & \multicolumn{1}{l|}{2}                 & \multicolumn{1}{l|}{0}               & 0             \\ \hline
        \multicolumn{1}{|l|}{HUILA}                    & \multicolumn{1}{l|}{332}               & \multicolumn{1}{l|}{228}             & 420           \\ \hline
        \multicolumn{1}{|l|}{LA GUAJIRA}               & \multicolumn{1}{l|}{127}               & \multicolumn{1}{l|}{24}              & 122           \\ \hline
        \multicolumn{1}{|l|}{MAGDALENA}                & \multicolumn{1}{l|}{0}                 & \multicolumn{1}{l|}{72}              & 48            \\ \hline
        \multicolumn{1}{|l|}{META}                     & \multicolumn{1}{l|}{110}               & \multicolumn{1}{l|}{22}              & 155           \\ \hline
        \multicolumn{1}{|l|}{NARIÑO}                   & \multicolumn{1}{l|}{177}               & \multicolumn{1}{l|}{239}             & 466           \\ \hline
        \multicolumn{1}{|l|}{NORTE DE SANTANDER}       & \multicolumn{1}{l|}{171}               & \multicolumn{1}{l|}{63}              & 422           \\ \hline
        \multicolumn{1}{|l|}{PUTUMAYO}                 & \multicolumn{1}{l|}{14}                & \multicolumn{1}{l|}{52}              & 19            \\ \hline
        \multicolumn{1}{|l|}{QUINDIO}                  & \multicolumn{1}{l|}{141}               & \multicolumn{1}{l|}{171}             & 144           \\ \hline
        \multicolumn{1}{|l|}{RISARALDA}                & \multicolumn{1}{l|}{141}               & \multicolumn{1}{l|}{67}              & 168           \\ \hline
        \multicolumn{1}{|l|}{SAN ANDRES Y PROVIDENCIA} & \multicolumn{1}{l|}{2}                 & \multicolumn{1}{l|}{6}               & 0             \\ \hline
        \multicolumn{1}{|l|}{SANTANDER}                & \multicolumn{1}{l|}{281}               & \multicolumn{1}{l|}{106}             & 852           \\ \hline
        \multicolumn{1}{|l|}{SUCRE}                    & \multicolumn{1}{l|}{39}                & \multicolumn{1}{l|}{0}               & 0             \\ \hline
        \multicolumn{1}{|l|}{TOLIMA}                   & \multicolumn{1}{l|}{287}               & \multicolumn{1}{l|}{167}             & 451           \\ \hline
        \multicolumn{1}{|l|}{VALLE DEL CAUCA}          & \multicolumn{1}{l|}{417}               & \multicolumn{1}{l|}{475}             & 468           \\ \hline
        \multicolumn{1}{|l|}{VICHADA}                  & \multicolumn{1}{l|}{5}                 & \multicolumn{1}{l|}{3}               & 0             \\ \hline
    \end{tabular}
    \caption{Cantidad de cultivos por departamentos}
    \label{tab:contingencia}
\end{table}

\begin{figure}[h!]
    \centering
    \begin{subfigure}{0.45\linewidth}
        \centering
        \includegraphics[width=\linewidth]{}
        \caption{Distribución por grupo etario}
        \label{fig:f1}
    \end{subfigure}
    \hfill
    \begin{subfigure}{0.45\linewidth}
        \centering
        \includegraphics[width=\linewidth]{}
        \caption{Nivel educativo más alto alcanzado}
        \label{fig:f2}
    \end{subfigure}
    \caption{Caracterización de la población encuestada}
\end{figure}

%Las \cref{fig:f3,fig:f4} ayudan a ilustrar esta información:

\begin{figure}[h!]
    \centering
    \begin{subfigure}{0.45\linewidth}
        \centering
        \includegraphics[width=\linewidth]{}
        \caption{Nivel de familiaridad con el término IA}
        \label{fig:f3}
    \end{subfigure}
    \hfill
    \begin{subfigure}{0.45\linewidth}
        \centering
        \includegraphics[width=\linewidth]{}
        \caption{Capacidad de dar ejemplos de IA}
        \label{fig:f4}
    \end{subfigure}
    \caption{Percepción de familiaridad y ejemplos de IA}
\end{figure}

\begin{figure}[h!]
    \centering
    \begin{subfigure}{0.45\linewidth}
        \centering
        \includegraphics[width=\linewidth]{}
        \caption{Confianza en los sistemas de IA}
        \label{fig:f5}
    \end{subfigure}
    \hfill
    \begin{subfigure}{0.45\linewidth}
        \centering
        \includegraphics[width=\linewidth]{}
        \caption{Percepción sobre el uso de datos por parte del gobierno y las empresas}
        \label{fig:f6}
    \end{subfigure}
    \caption{Confianza y percepción sobre uso de datos}
\end{figure}

% Conclusión
\section*{Conclusiones}
\end{document}
